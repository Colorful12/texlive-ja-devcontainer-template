\documentclass[a4paper, oneside, uplatex, 12pt]{book}

\usepackage[backend=biber,style=ieee]{biblatex}
\usepackage[dvipdfmx]{graphicx}
\usepackage{latexsym}
\usepackage{bmpsize}
\usepackage{url}
\usepackage{comment}
\usepackage{hyperref}
\usepackage{nameref} % for in-doc references | \labal{} |  \ref{}  \nameref{}

\def\Underline{\setbox0\hbox\bgroup\let\\\endUnderline}
\def\endUnderline{\vphantom{y}\egroup\smash{\underline{\box0}}\\}

\newcommand{\ttt}[1]{\texttt{#1}}

%% bibliography
\addbibresource{./src/bibliography.bib}


\begin{document}

%% use roman page numbering for abstract and acknowledgements
\pagenumbering{roman}

% \title{\bf{\LARGE{Sample of \LaTeX  Document} \\ \Large{\LaTeX のサンプルコード}}}
% \author{木幡 咲希\\早稲田大学}
% \date{28年28月28日(Tsu)}
% \maketitle

\begin{titlepage}
\begin{center}
    \vspace{0.1\textheight}
    {\Large Undergraduate Thesis 2022} \\
    \vspace{0.05\textheight}
    % \includegraphics[width=48truemm]{resources/0_title/waseda_logo.png} \\
    \vspace{0.05\textheight}
    \textbf{\huge TITLE!!!!!!!!!!!!!!!!!!!!!!!} \\
    \vfill
    {\Large Supervisor \hspace{0.02\textwidth} Edgar Simo-Serra} \\
    {\Large Area of Study \hspace{0.02\textwidth} Computer Graphics} \\
    \vspace{0.05\textheight}
    {\Large 
        Waseda University \\
        School of Fundamental Science and Engineering \\
        Department of Computer Science \\}
    \vspace{0.05\textheight}
    {\Large 1W192140-3 Saki Kohata \\}
    \vspace{0.05\textheight}
    {submitted on 20XX.0X.XX}
\end{center}
\end{titlepage}
    

%% abstract
\pagebreak
\hspace{0pt}
\vfill % magic commands to vertically center the content
    \begin{center}
    Abstract
    \end{center}
Pokémon is one of the most celebrated video game fran-
chise in history and competitive Pokemon battling has
earned increasing popularity as an esports over the last
decade. Competitive Pokemon battle poses a significant
challenge to developing AI agents due to partial observ-
ability, long-term causal relationship between states and
actions, an enormous knowledge base as well as the raw
complexity of the game system itself, which are also rep-
resentative of issues found in many real-world tasks. Self-
play reinforcement learning has shown promising results
in other well-studies games such as Go and Shogi, but we
struggled to achieve good performance with existing self-
play methods in Pokemon, as the agents often struggle to
get out of local optimums, possibly due to the partial ob-
servability and random nature of the game. To overcome
this, we introduce a novel multi-agent competition based
framework to train reinforcement learning agents to play
pokemon. Experiments showed that agents trained with
out method outperformed those trained with self-play,
even without reward shaping.
\vfill
\pagebreak

%% acknowledgement
\pagebreak
\hspace{0pt}
\vfill 
    \begin{center}
    Acknowledgements
    \end{center}
This thesis would not have been possible without the generous help from my
instructor Edgar Simo-Serra and support from PhD senior Kotaro Kikuchi. They
provided crucial technical support during the course of this research and has
been incredible friends. I am also deeply grateful to all other friends that
stayed together with me and provided mental supports during this long-lasting
pandemic.

\vfill
\pagebreak

%% use normal page numbering for the main body
\pagenumbering{arabic}

%% table of contents 
\tableofcontents
\listoffigures 
\listoftables

%% main body
\clearpage
\chapter{Introduction}
\section{Aim and Motivation}
 最後に書く!!!!!!
 


\section{Overview}
 This thesis consist of 6 chapters, including this introduction chapter Chapter 1.
In Chapter 2, we describe the various background knowledge and technologies that 
we draw upon in this paper, including a description of machine learning.
In Chapter 3, we introduce several papers that are related to our research.
In Chapter 4, we explain the proposed methodologies, and Chapter 5 details 
the experiment setup and presents the results.
Finally, in Chapter 6 we provide a summary of the main conclusions of 
the entire thesis.



\input{src/2_preliminaries}
\chapter{Related Works}

\section{Image Style Transfer}
 Gatys \textit{et al}. \cite{Gatys_2016_CVPR} introduced \textit{A Neural Algorithm 
of Artistic Style} which enables the creation of novel images that seamlessly blend 
the subject matter of a photograph with the artistic style of a famous artwork.
In rendering the semantic content of an image in various styles, the problem was
the lack of effective image representations that can explicitly encode semantic 
information and separate image content and style. 
\textit{A Neural Algorithm of Artistic Style} \cite{Gatys_2016_CVPR} is able to 
analyze the visual characteristics of an image and separate them into two distinct 
components: the content and the style.
The content refers to the subject or object depicted in the image, while the style 
refers to the visual techniques manifested in a painting, such as touch and mood.

The algorithm in the paper uses a convolutional neural network (CNN) to extract
content and style features from two input images. The goal is to generate an 
output image that minimizes the difference between the content and style features 
of the input and output images, as measured by two loss functions defined in the 
paper. 

To minimize the difference in content features between the input and output images,
it is necessary to consider reducing the content loss function. 
Let $\vec{p}$ and $\vec{x}$ be the input image and the image that is generated by
the model, and  $P^l$ and $F^l$ be their feature representations of these images in 
layer l of the CNN. Then the loss function of the content of images is defined as:

\begin{equation}
    \label{contentloss}
    \mathcal{ L}_{content}(\vec{p}, \vec{x}, l)=\frac{1}{2} \sum_{i, j}\left(F_{i j}^l-P_{i j}^l\right)^2
\end{equation}
Similarly, to minimize the difference in style features between the input and
output images, we consider reducing the style loss function. The style of an 
image can be represented by the correlations of the outputs of the filters in 
each layer. This feature correlation is given by a Gram matrix, which is 
expressed as:
\begin{equation}
    G_{i j}^l = \sum_k F_{i k}^l F_{j k}^l 
\end{equation}
Let $\vec{a}$ and $\vec{x}$ be the input image and the image that is generated by
the model, and $A^l$ and $G^l$ be their style representations of these images in 
layer l of the CNN. The contribution of layer l to the total loss can be 
expressed as:
\begin{equation}
    E_l=\frac{1}{4 N_l^2 M_l^2} \sum_{i, j}\left(G_{i j}^l-A_{i j}^l\right)^2
\end{equation}
Based on this, we can define the loss function for the style of images as:
\begin{equation}
    \label{styleloss}
    \mathcal{ L}_{style}(\vec{a}, \vec{x})=\sum_{l=0}^\mathcal{ L}w_{l}E_{l}
\end{equation}
To transfer the style of image $\vec{a}$ to image $\vec{p}$, it is needed to 
create a new image that maintains the content representation of $\vec{p}$ 
while adopting the style representation of $\vec{a}$. To achieve this, we must 
minimize both Equation \ref{contentloss} and Equations \ref{styleloss}.
The final loss function to minimize is:

\begin{equation}
    \mathcal{ L}_{total}(\vec{p}, \vec{a}, \vec{x})=\alpha\mathcal{ L}_{content}(\vec{p}, \vec{x})+\beta\mathcal{ L}_{style}(\vec{a}, \vec{x})
\end{equation}


Figure \ref{output_IST} shows the images generated using this algorithm.
A is the original image, photo from Andreas Praefcke.
In the lower left corner of B through F, the paintings that provided the style 
for each of the generated images are shown.
B : \textit{The Shipwreck of the Minotaur} by J.M.W. Turner, 1805. 
C : \textit{The Starry Night} by Vincent van Gogh, 1889. 
D : \textit{Der Schrei} by Edvard Munch, 1893. 
E : \textit{Femme nue assise} by Pablo Picasso, 1910. 
F : \textit{Composition VII} by Wassily Kandinsky, 1913.

These are high quality paintings to which the style, such as the touch of the 
brush, has been adapted. 
However, the colors of the output picture are those of the style image, 
not the original image, and the result images have the characteristic objects of 
the style image (e.g., the moon, stars, and wavy clouds in \textit{The Starry Night}).
In this paper, we aim to address the problem of transferring only brush style 
in paintings.

\begin{figure}[p]
    \centering
    \includegraphics[width=130truemm]{resources/3_related_work/outputs_IST.png}
    \caption{
        Examples of images combining content and style, taken from
        \cite{Gatys_2016_CVPR}. 
    }
    \label{output_IST}
\end{figure}
\clearpage

\section{Machine painting like a human painter : DRL}
 Painting is a form of artistic expression that usually involving imaginative or 
creative skills, and it takes a lot of time and proper training to master the 
art.  Teaching a machine to paint is a challenging task, but researching it 
could lead to the development of tools that assist people in painting.

Huang \textit{et al}. \cite{Huang_2019_ICCV} trained a painting agent which 
imitates the painting process of humans. 
They employ a neural renderer in model-based deep reinforcement learning.
The agent is trained using model-based reinforcement learning, in which the 
agent learns to predict the outcomes of its actions and uses these predictions 
to plan its next actions, such as selecting a brush or color.
The agent is able to receive a larger reward as the visual quality of the 
paintings it create is better. The agent learns to make good painting choices by 
trying out various actions with the goal of maximizing the reward it receives.

It is demonstrated that their agent is able to learn to paint a variety of 
different styles and subjects, including handwritten digits, landscapes, 
portraits, and natural scene images. (Figure \ref{LearnToPaint})
\begin{figure}[h]
    \centering
    \includegraphics[width=120truemm]{resources/3_related_work/LearnToPaint.png}
    \caption{
        The painting results generated from the model introduced in the paper,
        taken from \cite{Huang_2019_ICCV}.
        (a) : MNIST dataset from \cite{deng2012mnist}. (b) SVHN dataset from \cite{netzer2011reading}.
        (c) : CelebA dataset from \cite{Liu_2015_ICCV}. 
        (d) : ImageNet dataset from \cite{russakovsky2015imagenet}.
    }
    \label{LearnToPaint}
\end{figure}
\newline
As shown in Figure \ref{LearnToPaint}, the model-based deep reinforcement 
learning method generates attractive paintings, but there seems to be problems
such as long training time for the agent and the instability of the agent.

\section{Machine painting like a human painter : FNNs}
 Liu \textit{et al}. \cite{liu2021paint} proposed Paint Transformer, a novel 
framework paintings from natural images by predicting parameters of multiple 
strokes without using reinforcement learning. 
Paint Transfer is a Transformer-based framework. This framework generates 
painting from natural images by using a feed-forward transformer to predict 
the multiple stroke parameters. 
Paint Transformer consists of two modules: the Stroke Predictor and the Stroke Renderer.  

In this work, strokes are represented by shape parameters ${x, y, h, w, \theta}$ 
and color parameters ${r, g, b}$.
\begin{figure}[h]
    \centering
    \includegraphics[width=110truemm]{resources/3_related_work/stroke-params.png}
    \caption{
        Parameter definition of a stroke,
        taken from \cite{liu2021paint}.
    }
    \label{strokeparams}
\end{figure}
\newline
The parameters are defined as shown in Figure \ref{strokeparams}, the center 
coordinates ($x$ and $y$), the height ($h$) and width ($w$) of the stroke, 
and the angle of rotation ($\theta$). Additionally, the RGB values of the 
stroke are represented by $r$, $g$, and $b$. Thus, the strokes generated are 
limited to monochromatic and linear.

Paint Transformer pipeline is shown in Figure \ref{PTpipeline}.
Stroke Predictor takes the target image It and the intermediate 
image $I_c$ as input, and generates a set of strokes $S_r$
(strictly speaking, it generates the parameters to determine the stroke set $S_r$).
The goal of training the Stroke Predictor is to make $S_r$ match a reference set
of strokes $S_f$. 
Stroke Renderer then takes each stroke in $S_r$ and generates a stroke image, 
which is combined with $I_c$ to create the final result image $I_r$.
This process can be expressed as follows:
\begin{equation}
    \label{painttransformer}
    I_r = PaintTransformer(I_c, I_t)
\end{equation}
\vspace{5mm}
\begin{figure}[t]
    \centering
    \includegraphics[width=150truemm]{resources/3_related_work/PTpipeline.png}
    \caption{
        A pipeline for Paint Transformer,
        taken from \cite{liu2021paint}.
    }
    \label{PTpipeline}
\end{figure}

\vspace{0.7cm}

\begin{figure}[h]
    \centering
    \includegraphics[width=150truemm]{resources/3_related_work/PTresult.png}
    \caption{
        The painting results generated from Paint Transformer,
        taken from \cite{liu2021paint}.
    }
    \label{PTresult}
\end{figure}
As shown in Figure \ref{PTresult}, this model produces highly quality paintings.
However, the strokes used are currently limited to monochromatic and linear forms.
To generate paintings with various touches, it should be explored more complex 
strokes with various shapes or color patterns.


\input{src/4_methodology}
\chapter{Experiments and Results}

\section{Experimental Settings}
 Our experiments are conducted using the environment shown in Table \ref{environment}.

\setlength{\tabcolsep}{6pt}
\begin{table*}[h]
\begin{center} 
\caption{Environment of experiments}
\label{environment}
{\normalsize
\begin{tabular}{l c}

% content goes here
\toprule
Device & Available resources \\
\midrule
CPU & Intel(R) Xeon(R) E5-2690v4 CPU (2.60GHz)\\
Memory size & DDR4-2400 Reg.ECC 32GB\\
GPU & GeForce GTX 1080Ti  \\
\bottomrule
\end{tabular}
} \end{center} \end{table*}
\setlength{\tabcolsep}{6pt}


\section{Experiments}
 The model takes a content image and a style image and outputs a result image.
For the content images, we used one cat and one dog image from \textit{The Oxford-IIIT Pet Dataset}
\cite{Oxford-data} .
The content images and style images we used are shown in Figure \ref{dataset}.
\begin{figure}[h]
    \centering
    \includegraphics[width=115truemm]{resources/5_e_and_r/dataset.pdf}
    \caption{
        Content images and style images.
    }
    \label{dataset}
\end{figure}
Image (c) is \textit{Piet Mondrian Composition with Red, Blue, and Yellow}
by Piet Mondrian, 1930.
Image (d) is \textit{The Starry Night} by Vincent van Gogh, 1889. 

\vspace{3cm}
\section{Results}
 The output images obtained when painting the images (a) and (b) in Figure \ref{dataset} 
using all the brushes shown in Figure \ref{Brushes} are shown in Figure \ref{results}.
Images (a) to (g) in Figure \ref{results} correspond to brushes (a) to (g) in Figure \ref{Brushes}.
When we set images (c) and (d) in Figure \ref{dataset} as input to this 
model as style images, the images selected as output were images (d) and (e) 
in Figure \ref{results}, respectively.
This means that the brush style of image (c) in Figure \ref{dataset} and image 
(d) in Figure \ref{results} are considered similar. The same is true for image 
(d) in Figure \ref{dataset} and image (e) in Figure \ref{results}.
For clarity, the style reference image and the resulting image are shown side 
by side in Figure \ref{compare}.

 The generated image does not perfectly mimic the brush style of the reference 
image. 
It is considered that this is attributed to the fact that there are only seven 
base brushes and a minimal number of parameters to represent strokes.

\begin{figure}[p]
    \centering
    \includegraphics[width=90truemm]{resources/5_e_and_r/results.pdf}
    \caption{
        The output images.
    }
    \label{results}
\end{figure}

\begin{figure}[h]
    \centering
    \includegraphics[width=130truemm]{resources/5_e_and_r/compare.pdf}
    \caption{
        The style reference images input to our model and the resulting 
        images generated by the model.
    }
    \label{compare}
\end{figure}
\chapter{Discussions}

ブラシ画像は同じ画像をひっぱってつくられているけど、同じ画像の繰り返しにする
パターンもあって良いと思う。(クリスタのブラシで例を見せる)




% \section{はじめに}
% \begin{figure}[t]
%     \begin{center}
%         \includegraphics[width=7cm]{image/syokuji_computer.png}
%         \caption{パソコンの前でご飯を食べる人のイラスト}
%         \label{fig:syokuji_computer}
%     \end{center}
% \end{figure}

% パソコンの前でご飯を食べることはよくある。パソコンの前でご飯を食べる人のイラストを図\ref{fig:syokuji_computer}に示す。
% このイラストは、規約の範囲内であれば、個人、法人、商用、非商用問わず無料で利用できることでおなじみの、{\bf かわいいフリー素材 いらすとや}\cite{irasutoya}より引用した。
% \cite{liu2021paint}

% \section{おわりに}
% やっぱり{\bf いらすとや}のイラストはすばらしい。


%% bib
\printbibliography


\end{document}
