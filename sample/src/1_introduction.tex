\chapter{Introduction}
\section{Aim and Motivation}
 Humans have found value in expressing their emotions and thoughts through 
painting. Painting is considered the best way to enhance human imagination. 
Before the advent of digital illustration, painting was considered something 
that was difficult to start as it required tools and supplies such as paint 
and brushes. With the development of digital illustration, anyone can easily 
paint by obtaining the necessary hardware and software.
However, to create great works, it is necessary to learn about composition, 
sketching techniques, color and lighting effects, and this can take a long 
time to master. Recently, the technology of computer-generated art has also 
developed, allowing many people to easily generate their own paintings. For 
example, Midjourney \cite{midjourney} is a software that can generate great 
paintings by simply inputting the desired image or keywords. Midjourney is 
popular because it is easy to use, and the generated paintings are high quality. 
Even professional painters have a good reputation for it.
However, for painters, regardless of whether it is digital or analog, it is 
important to create paintings with their own hands. We would like to conduct 
research that will lead to support for creators. 
We set our goal  to develop a model that utilizes two input images 
- one as a content reference and the other as a style reference - 
and generates an output image that captures the semantic content of the content 
reference while applying the brushstrokes of the style reference. 
Figure \ref{haru-image} shows an example of the two input images and the output image.
\begin{figure}[h]
    \centering
    \includegraphics[width=130truemm]{resources/1_introduction/haru_image.pdf}
    \caption{
        What we want to create.
    } 
    \label{haru-image}
\end{figure}
We consider that by using a simple brush to manually create an image like the 
content image shown in Figure \ref{haru-image}, and then relying on the computer 
to generate an image with the brush style of the style image shown in Figure 
\ref{haru-image}, the painting process could be significantly streamlined for 
artists.

Recent advancements in AI technology have enabled the transformation of natural 
images into those with an artistic style, and research on transferring styles 
from one image to another has already been conducted \cite{Gatys_2016_CVPR}.  
the models proposed in the study is aimed at transferring the atmosphere of the 
image, and they also transfer the distinctive objects depicted in the style 
reference image. 
Our objective is to transfer the brush style while keeping the content image 
unchanged. In this paper, we utilized a Transformer-based framework to generate 
strokes that imitate the brush style of the reference image. 


\section{Overview}
 This thesis consist of 6 chapters, including this introduction chapter Chapter 1.
In Chapter 2, we describe the various background knowledge and technologies that 
we draw upon in this paper, including a description of machine learning.
In Chapter 3, we introduce several papers that are related to our research.
In Chapter 4, we explain the proposed methodologies, and Chapter 5 details 
the experiment setup and presents the results.
Finally, in Chapter 6 we provide a summary of the main conclusions of 
the entire thesis. Our proposed method didn't perfectly imitate the brush style 
of the reference image. However, we believe that the performance of this method 
can be improved by increasing the number of parameters used to control brush 
strokes.


