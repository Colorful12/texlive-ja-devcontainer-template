% MIRU2023: How to use the LaTeX class Ver.1
%
\documentclass[MIRU,submit,english]{miru2023e}
\usepackage[dvipdfmx]{graphicx}
%\usepackage{latexsym}
%\usepackage[fleqn]{amsmath}
%\usepackage[psamsfonts]{amssymb}

\begin{document}

\title{How to Use \LaTeXe\ Class File for MIRU2023}

\affiliate{Tokyo}{First University}
\affiliate{Osaka}{Second University (Presently with Third Corporation)}

 \author{Hanako GAZO}{Tokyo}[hanako@gazo.ac.jp]
 \author{Taro NINSHIKI}{Osaka}[taro@ninshiki.co.jp]
 \author{Jiro RIKAI}{Osaka}[jiro@rikai.co.jp]
 \author{John Smith}{Osaka,Tokyo}[john@rikai.co.jp]

\maketitle

\section*{Abstract}
Write an abstract in about 200 words.

\section{Introduction}
This is a sample of MIRU2023 papers.
The paper is compiled using p\LaTeXe. 

\section{How to prepare extended abstract}

\subsection{Language}

Either Japanese or English is acceptable. 

\subsection{Paper length}

The maximum length of a paper is four pages \textcolor{red}{(excluding references)}. 

\subsection{Author' names and affiliations}

Please list the full names, affiliations, and email addresses of the authors.
(The review process of selecting oral papers is single-blind.)

\section{Main text}

\begin{figure}[h]
    \centering
    \includegraphics[width=.3\linewidth]{logo.pdf}
    \caption{Caption for PDF}
    \label{fig:my_label1}
\end{figure}

\begin{figure}[h]
    \centering
    \includegraphics[width=.3\linewidth]{logo.png}
    \caption{Caption for PNG}
    \label{fig:my_label2}
\end{figure}

\begin{figure}[h]
    \centering
    \includegraphics[width=.3\linewidth]{logo.jpg}
    \caption{Caption for JPEG}
    \label{fig:my_label3}
\end{figure}

%\input{example_paper_text_en.tex}

%\bibliographystyle{miru2022e}
%\bibliography{myref}

\begin{thebibliography}{9}% If the number of articles and books is more than 9, use {99} instead of {9}.
\bibitem{1}
W. Rice, A. C. Wine, and B. D. Grain,
diffusion of impurities during epitaxy,
Proc. IEEE, vol.~52, no.~3, pp.~284--290, March 1964.
\bibitem{2}
 H. Tong, Nonlinear Time Series: A Dynamical System Approach, J. B. Elsner, ed., Oxford University Press, Oxford, 1990.
\bibitem{3}
H. K. Hartline, A. B. Smith, and F. Ratlliff,
Inhibitoryinteraction in the retina,
in Handbook of Sensory Physiology,
ed. M. G. F. Fuortes, pp.~381--390, Springer-Verlag, Berlin.
\bibitem{4}
Y. Yamamoto, S. Machida, and K. Igeta,
``Micro-cavity semiconductors with enhanced spontaneous emission,''
Proc. 16th European Conf. on Opt. Commun.,
no.~MoF4.6, pp.~3--13, Amsterdam, The Netherlands, Sept.1990.
\end{thebibliography}

\end{document}
