\chapter{Background}
\section{Machine Learning}

 Machine learning is a method of analyzing data in which a 
computer automatically iterates through large amounts of data 
without explicit programming to find patterns hidden in the data. 
The purpose of machine learning is to improve computer performance 
on a specific task over time through learning from data.
In other words, machine learning allows computers to adapt 
and improve performance on their own. 
Machine learning algorithms can be classified into two main 
categories: supervised learning and unsupervised learning. 
 
教師あり学習、なし学習の例、と説明
kaggleのtitanicを教師あり学習の例にしようかな。
教師ありではデータセットが精度を左右する。実用的な精度を達成するには、
十分なデータセットが必要。

\section{Neural Networks}
機械学習の一種で、人間の神経回路を真似することで学習を実現。
元祖はフランクローゼンブラっとが提唱した単純パーセプトロンという
ニューラルネットワークである。NNを多層にしたものをディープラーニング
という。
次に説明する CNNはニューラルネットワークの一種である的な。
\section{Convolutional Neural Network}
CNNについての説明。
畳み込みとかプーリングとかはここで説明する。

\section{Transformers}
あああああああああああ

\section{Autoencoders} 
ああああああああああああ