% MIRU2023 LaTeXクラスファイルの使い方 Ver.1
\documentclass[MIRU,submit,uplatex]{miru2023j}
\usepackage[dvipdfmx]{graphicx}
%\usepackage{latexsym}
%\usepackage[fleqn]{amsmath}
%\usepackage[psamsfonts]{amssymb}

\begin{document}

\title{MIRU2023 \LaTeXe\ クラスファイルの使い方}

\affiliate{Tokyo}{第一大学}
\affiliate{Osaka}{第二大学(現在,第三コーポレーション勤務)}

 \author{画像 花子}{Hanako GAZO}{Tokyo}[hanako@gazo.ac.jp]
 \author{認識 太郎}{Taro NINSHIKI}{Osaka}[taro@ninshiki.co.jp]
 \author{理解 次郎}{Jiro RIKAI}{Osaka}[jiro@rikai.co.jp]
 \author{John Smith}{John Smith}{Osaka,Tokyo}[john@rikai.co.jp]


\maketitle

\section*{概要}
本稿は,MIRU2023用のサンプルです.最初に200文字程度の概要を記述してください.

\section{はじめに}
本稿はMIRU2023用の原稿サンプルです.
p\LaTeXe\ に基づいて作成しています.

\section{原稿の書き方}

\subsection{言語}
使用言語は,日本語または英語です.

\subsection{ページ数}
ページ数は\textcolor{red}{参考文献を除いて}最大4ページです.

\subsection{著者名と所属}
著者名と所属,連絡先メールアドレスを原稿に記載して下さい
(口頭発表候補論文の評価方法はシングルブラインドです).

\subsection{キーワード}
キーワードは省略して下さい.

\section{本文}

\begin{figure}[h]
    \centering
    \includegraphics[width=.3\linewidth]{logo.pdf}
    \caption{Caption for PDF}
    \label{fig:my_label1}
\end{figure}

\begin{figure}[h]
    \centering
    \includegraphics[width=.3\linewidth]{logo.png}
    \caption{Caption for PNG}
    \label{fig:my_label2}
\end{figure}

\begin{figure}[h]
    \centering
    \includegraphics[width=.3\linewidth]{logo.jpg}
    \caption{Caption for JPEG}
    \label{fig:my_label3}
\end{figure}

%\input{example_paper_text_jp.tex}

%\bibliographystyle{miru2022j}
%\bibliography{myref}

\begin{thebibliography}{9}% 文献数が10未満の時 {9}
\bibitem{1}
W. Rice, A. C. Wine, and B. D. Grain,
diffusion of impurities during epitaxy,
Proc. IEEE, vol.~52, no.~3, pp.~284--290, March 1964.
\bibitem{2}
 H. Tong, Nonlinear Time Series: A Dynamical System Approach, J. B. Elsner, ed., Oxford University Press, Oxford, 1990.
\bibitem{3}
H. K. Hartline, A. B. Smith, and F. Ratlliff,
Inhibitoryinteraction in the retina,
in Handbook of Sensory Physiology,
ed. M. G. F. Fuortes, pp.~381--390, Springer-Verlag, Berlin.
\bibitem{4}
Y. Yamamoto, S. Machida, and K. Igeta,
``Micro-cavity semiconductors with enhanced spontaneous emission,''
Proc. 16th European Conf. on Opt. Commun.,
no.~MoF4.6, pp.~3--13, Amsterdam, The Netherlands, Sept.1990.
\end{thebibliography}

\end{document}
