\chapter{Experiments and Results}

\section{Experimental Settings}
 Our experiments are conducted using the environment shown in Table \ref{environment}.

\setlength{\tabcolsep}{6pt}
\begin{table*}[h]
\begin{center} 
\caption{Environment of experiments}
\label{environment}
{\normalsize
\begin{tabular}{l c}

% content goes here
\toprule
Device & Available resources \\
\midrule
CPU & Intel(R) Xeon(R) E5-2690v4 CPU (2.60GHz)\\
Memory size & DDR4-2400 Reg.ECC 32GB\\
GPU & GeForce GTX 1080Ti  \\
\bottomrule
\end{tabular}
} \end{center} \end{table*}
\setlength{\tabcolsep}{6pt}


\section{Experiments}
 The model takes a content image and a style image and outputs a result image.
For the content images, we used one cat and one dog image from \textit{The Oxford-IIIT Pet Dataset}
\cite{Oxford-data} .
The content images and style images we used are shown in Figure \ref{dataset}.
\begin{figure}[h]
    \centering
    \includegraphics[width=115truemm]{resources/5_e_and_r/dataset.pdf}
    \caption{
        Content images and style images.
    }
    \label{dataset}
\end{figure}
Image (c) is \textit{Piet Mondrian Composition with Red, Blue, and Yellow}
by Piet Mondrian, 1930.
Image (d) is \textit{The Starry Night} by Vincent van Gogh, 1889. 

\vspace{3cm}
\section{Results}
 The output images obtained when painting the images (a) and (b) in Figure \ref{dataset} 
using all the brushes shown in Figure \ref{Brushes} are shown in Figure \ref{results}.
Images (a) to (g) in Figure \ref{results} correspond to brushes (a) to (g) in Figure \ref{Brushes}.
When we set images (c) and (d) in Figure \ref{dataset} as input to this 
model as style images, the images selected as output were images (d) and (e) 
in Figure \ref{results}, respectively.
This means that the brush style of image (c) in Figure \ref{dataset} and image 
(d) in Figure \ref{results} are considered similar. The same is true for image 
(d) in Figure \ref{dataset} and image (e) in Figure \ref{results}.
For clarity, the style reference image and the resulting image are shown side 
by side in Figure \ref{compare}.

 The generated image does not perfectly mimic the brush style of the reference 
image. 
It is considered that this is attributed to the fact that there are only seven 
base brushes and a minimal number of parameters to represent strokes.

\begin{figure}[p]
    \centering
    \includegraphics[width=90truemm]{resources/5_e_and_r/results.pdf}
    \caption{
        The output images.
    }
    \label{results}
\end{figure}

\begin{figure}[h]
    \centering
    \includegraphics[width=130truemm]{resources/5_e_and_r/compare.pdf}
    \caption{
        The style reference images input to our model and the resulting 
        images generated by the model.
    }
    \label{compare}
\end{figure}